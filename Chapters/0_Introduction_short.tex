% Chapter 1
\addchap{Въведение}
\label{chapter-introduction} 
%----------------------------------------------------------------------------------------
% Define some commands to keep the formatting separated from the content 
\newcommand{\keyword}[1]{\textbf{#1}}
\newcommand{\tabhead}[1]{\textbf{#1}}
\newcommand{\code}[1]{\texttt{#1}}
\newcommand{\file}[1]{\texttt{\bfseries#1}}
\newcommand{\option}[1]{\texttt{\itshape#1}}
%----------------------------------------------------------------------------------------
\section* {Значимост на темата}
\addcontentsline{toc}{section}{Значимост на темата}
Необходимостта от създаване на интегрирана информационна система, обслужваща нуждите на учените, занимаващи се с биологично разнообразие, датира най-малко от 1985 г., когато е създадена Работна група по таксономични бази-данни (TDWG). В последствие групата е преименувана на Стандарти за информационни технологии за биологичното разнообразие, но запазва съкращението TDWG\footnote{Уеб-страница с историята на TDWG, започваща от 1985 г., може да се види на \url{http://old.tdwg.org/past-meetings/}. Много от препратките за съжаление са невалидни и страницата се нуждае от известна поддръжка.}. През 1999 г. е създаден Глобален институт за информация за биологичното разнообразие (GBIF), след като Организацията за икономическо сътрудничество и развитие (ОИСР) стига до извода, че е необходим международен механизъм за достъп до данни и информация за био-разнообразието в световен мащаб (\cite{noauthor_what_nodate}). Декларацията Bouchout (\cite{noauthor_bouchout_2014}) ознаменува резултатите от финансирания от Европейския Съюз проект \cite {noauthor_pro-ibiosphere_nodate}, който продължава от 2012 до 2014 г. и е посветен на задачата за създаване на интегрирана информационна система за биологичното разнообразие. Декларацията Bouchout призовава към свободното предоставяне на научни знания за биологичното разнообразие като Linked Open Data (LOD). Успореден процес в САЩ започва още по-рано със създаването на Архитектура за глобални имена, GNA (\cite{patterson_names_2010, pyle_towards_2016}).

През 2014 г., в рамките на европейската програма ``Хоризонт 2020'', между академични институции и частни фирми е създаден консорциумът BIG4. BIG4 цели да допринесе за напредъка на науката за биологичното разнообразие. Мисията на проекта е: ``BIG4 - Биосистематика, информатика и генетиката на големите 4 групи насекоми: обучение учените и предприемачите от утрешния ден'' (\cite{university_of_copenhagen_big4_2014}). Важен член на консорциума е академичното издателство и софтуерна компания Пенсофт. Пенсофт публикува няколко десетки добре познати таксономични списания с отворен достъп\footnote{Например, ZooKeys, PhytoKeys, MycoKeys и Biodiversity Data Journal (BDJ).}. Като подписваща страна на декларацията Bouchout, Pensoft е естествен кандидат за  изграждане на визията  за Отворена система за управление на знанията за биоразнообразието (OBKMS). Този дисертационнен проект за придобиване на докторска степен е базиран в академично издателство Пенсофт и в Института по информационни и комуникационни технологии (IICT) на Българската академия на науките.

\section* {Преглед на литературата}
\addcontentsline{toc}{section} {Преглед на литературата}
Поради интердисциплинарния характер на дисертацията този раздел ще се фокусира върху преглед на две области: (а) Бази от знания и Свързани данни, а също така и върху (б) Публикуване на данни в сферата на биоразнообразието.

\subsection*{Бази от знания и  Свързани данни}
\addcontentsline{toc}{subsection}{Бази от знания и Свързани данни}
Ще започнем с въвеждането на термините \emph{база от знания (knowledge base)} и \emph {система, използваща знание (knowledge-based system)}. В дисертацията двата термина се използват взаимозаменяемо, но сме склонни да изписваме по-дългия вариант, система, използваща знания, когато искаме да подчертаем аспекти на базата от знания, които не са свързани с технологичните аспекти на базата-данни, съхраняваща знанието. Ще дефинираме система, използващи знание, като разгледаме изрични дефиниции, както и като разгледаме реализациите на няколко такива системи на практика. Терминът е широко обсъждан още през 80-те години на 20-ти век (\cite{brodie_kbms_1989}) и началото на деветдесетте години (\cite{harris_knowledge_1993}), като значението, влагано му тогава, е ``използване на идеи, както от системите за управление на бази-данни (DBMS), така и от изкуствения интелект за създаване на тип компютърна система, наречена \emph {knowledge management system} (KBMS)''. \cite{harris_knowledge_1993} пише, че характеристиките на KBMS са, че тя съдържа ``предписани правила и факти, от които полезни изводи могат да бъдат извлечени от машина за правене на извод (inference engine)''. Трябва да отбележим, че това тълкуване идва от времето на AI системите от първо поколение, които са базирани на правила (rule-based systems). В последно време бе постигнат напредък в включването на статистически техники в базите-данни (\cite {mansinghka_bayesdb:_2015}). Въпреки това, в този проект работим с класическото разбиране за KBMS, основано на логически правила. С други думи под база от знания разбираме ``подходяща база данни, тясно интегрирана с логически слой, който позволява правенето на логичиски изводи и придава семантика на данните.''

Друго относително по-ново развитие в системите, използващи знание, е приложението на принципите на Свързаните данни (\cite{heath_linked_2011}). Повечето съществуващи бази от знания наблягат на социални аспекти, които правят данните взаимосвързани и годни за повторно използване. Примери са Freebase (\cite{bollacker_freebase:_2008}), която неотдавна беше включена в WikiData (\cite{vrandecic_wikidata:_2014, pellissier_tanon_freebase_2016}), DBPedia (\cite {hutchison_dbpedia:_2007}), както и Wolfram|Alpha (\cite{noauthor_wolfram|alpha_nodate}) и Google Knowledge Graph (\cite {singhal_introducing_2012}). Общото между тези системи е, че акцентът се поставя не само върху логическия слой, който позволява изводи, а върху единното информационно пространство: тези системи действат като интегрират информация от множество места и следват в различни степени принципите на Свързаните отворени данни, Linked Open Data (LOD). Свързаните отворени данни (\cite{heath_linked_2011}) са списък от препоръки на Семантичната мрежа (Semantic Web, \cite{berners-lee_semantic_2001}), които, когато се прилагани правилно, гарантират, че публикуваните в мрежата данни са годни за използване от всички потребители на мрежата. Ще разгледаме подробно принципите на Свързаните данни и тяхното приложение към OpenBiodiv в глава~\ref{chapter-lod}.

Водени от тези тенденции, модерните бази от знания наблягат повече върху свързването на данни, отколкото върху разработването на сложни механизми за логически изводи. Налице е критика на идеята за интегриране на логически слой в базата-данни, тъй като това интегриране води до повишена сложност (\cite {barrasa_rdf_2017}). Критиката може да бъде обобщена с две точки. Първо, поставянето на логика близо до данните (особено когато е прекомерно мощна за задачата) може да доведе до драстично намаляване на производителността\footnote {Ще сравним ефективността на логическия слой на мощния онтологичен език (OWL) с по-слабата RDF схема (RDFS)  в глава~\ref{chapter-lod}.}.  Второ, нови техники (например машинно обучение) могат да направят съществуващия логически слой излишен. Нашата гледна точка е, че данните са обектът, който е много по-ценен, докато стратегията за вадене на изводи (дали е логически слой, основан на правила, или техниката за статистическо машинно обучение) може да бъде подменяна с напредването на изчислителните науки. Тези идеи водят до интересен главоблъсък в избора на технология за база-данни, разгледана в следващите раздели.

И накрая, базирана на знания система трябва да включва задължително и компоненти на потребителския интерфейс и/или интерфейс за програмиране на приложения (API), както и приложения (apps). Те служат като точка на контакт между човека и машината и са от решаващо значение за успеха на всяка такава система.

\subsection*{Публикуване на данни за биоразнообразието}
\addcontentsline{toc}{subsection}{Публикуване на данни за биоразнообразието}
В био-медицинската област отдавна се работи за извличане на информация и откриване на знания от първична литература (напр. \cite {rebholz-schuhmann_facts_2005, momtchev_expanding_2009, williams_open_2012}). Областта на биоразнообразието, и по-специално биологичната систематика и таксономия (от тук нататък в дисертацията съкратено наричана \emph{таксономия}), също се движи в посока към семантизация (напр. \cite{agosti_biodiversity_2006, patterson_taxonomic_2006, kennedy_scientific_2005, penev_fast_2010, tzitzikas_integrating_2013}). Академичната издателска дейност е моделирана чрез Онтологичните издателски и референтни онтологии, SPAR Ontologies (\cite {peroni_semantic_2014}). Онтологиите на SPAR са колекция от онтологии, включващи, наред с другото, библиографската онтология FaBiO (\cite{peroni_fabio_2012}) и DoCO, онтология за компонентите на даден документ (\cite{constantin_document_2016}). Онтологиите на SPAR осигуряват набор от класове и свойства за описанието на академични статии с общо предназначение. Таксономичните статии и техните компоненти, от друга страна, са моделирани чрез TaxPub XML Document Type Definition (DTD), която ще наричаме свободно XML схема (\cite{catapano_taxpub:_2010}). TaxPub е XML схема за таксономично публикуване на няколко важни таксономични списания (напр. ZooKeys, PhytoKeys, Biodiversity Data Journal) и служи като концептуален шаблон за OpenBiodiv-O (глава~\ref{chapter-ontology}).

Таксономичната номенклатура е дисциплина с много дълга традиция. Тя се трансформира в модерната си форма с публикуването на линеевата система (\cite {linnaeus_systema_1758}). Вече към началото на миналия век са били използвани стотици таксономични термини (\cite{witteveen_naming_2015}). Понастоящем именуването на групи от организми се регулира от Международния кодекс за зоологична номенклатура (ICZN) (\cite{international_commission_on_zoological_nomenclature_official_2017}) и Международния кодекс за номенклатура на водорасли, гъби и растения (Кодекс на Мелбърн) (\cite{noauthor_international_2012}). Поради тяхната сложност (напр. ICZN има 18 глави и 3 приложения), се оказва предизвикателство да се създаде онтология базиране на законите на биологичната номенклатура. Oпити включват сравнително пълната онтология на NOMEN (\cite{dmitriev_nomen_2017}) и не до там завършените Термини за статута на таксономичните номенклатури TNSS\footnote{Въпреки че не е известно на авторите, дали TNSS е публикуван в рецензиранaта литература, остатъци от нея все още могат да се намерят на GitHub, напр. под \url{https://github.com/pensoft/OpenBiodiv/blob/master/ontology/contrib/taxonomic_nomenclatural_status_terms.owl}}.

Има няколко проекта, които целят моделиране на по-широката област на биологичното разнообразие. Дарвин за Семантичната мрежа, Darwin-SW (\cite{baskauf_darwin-sw:_2016}) адаптира предишните термини на DarwinCore (\cite{wieczorek_darwin_2012}) като Resource Description Framework (RDF). Тези модели се занимават основно с данни за наличие на организми. Моделирането и формализирането на строго таксономичния домейн бе обсъдено от \cite{berendsohn_concept_1995} и по-късно в напр. \cite{franz_perspectives:_2009, sterner_taxonomy_2017}. Забележителни усилия са XML-базираната схема за прехвърляне на таксономични концепции (\cite{taxonomic_names_and_concepts_interest_group_taxonomic_2006}) и вече невалидната Taxon Concept Ontology. Съвсем наскоро общността на TDWG се опита да възкреси онтологията на таксономичните концепции със създаването на Група за имена и таксономични концепции. Груповите дискусии могат да бъдат достъпнени под \url{https://github.com/tdwg/tnc}. Интересното е, че \href {https://github.com/tdwg/tnc/issues/1} {в първата дискусия в GitHub} се обсъди OpenBiodiv-O и възможността за приемането му като стандарт TDWG.

Към  юни 2015 г., когато OpenBiodiv беше започнат, бяха публикувани някои статии  по темите за свързване на данни и споделяне на идентификатори в в областта биологичното разнообразие (\cite{page_biodiversity_2008}), относно обединяването на филогенетични знания (\cite{parr_evolutionary_2012}), и по темата на таксономичните имена и връзката им със семантичната мрежа (\cite {page_taxonomic_2006, patterson_names_2010}), както и относно обобщаването и изследванията на биологичното разнообразие (\cite{mindell_aggregating_2011}). Дискусия на OBKMS може да се намери в научния блог \href {http://iphylo.blogspot.bg} {iPhylo} (\cite {page_vision_2014, page_putting_2015}). Правните аспекти на OBKMS бяха обсъдени от \cite {egloff_open_2014}. Освен това няколко системи за интегриране на данни за биоразнообразието бяха разработени от различни групи. Някои от най-важните са UBio, Глобални имена, BioGuid, BioNames, Профил на таксони на Pensoft и Plazi\footnote{UBio: \href{http://ubio.org/}{http://ubio.org/}; Глобални имена: \href {http://globalnames.org/}{http://globalnames.org/}; BioGuid: \href {http://bioguid.org/}{http://bioguid.org/}; BioNames: \href{http://bionames.org/}{http://bionames.org/}; Профил на таксон на Pensoft: \href {http://ptp.pensoft.eu/}{http://ptp.pensoft.eu/}; Plazi: \ href {http://plazi.org/wiki/} {http://plazi.org/wiki/}}.

\subsection*{Основни констатации}
\addcontentsline{toc}{subsection}{Основни констатации}

Основните констатации от цитираните в предишните параграфи източници могат да бъдат обобщени, както следва:

\begin{enumerate}

\item{Биоразнообразието се занимава с различни видове данни: таксономични, биогеографски, филогенетични, визуални, описателни и други. Тези данни са скрити в несвързани хранилища за данни.}

\item{Базите данни за биологичното разнообразие се нуждаят от универсална система за именуване на концепции поради недостатъците на линейските имена за модерната таксономия. Етикитете за таксономични концепции са предложени като четимо от човека решение и Глобално-стабилни уникални идентификатори (GUID) на таксономични концепции като четимо от машината решение.}

\item{Налице е основа от цифровизирана полуструктурирана информация за биологичното разнообразие в мрежата с подходящи лицензи, чакащи да бъдат интегрирани като база от знания.}

\end{enumerate}


\section*{Цел и задачи}
\addcontentsline{toc}{section} {Цел и задачи}

Предвид огромния международен интерес към OBKMS, чрез тази дисертация се стартира проектът OpenBiodiv, чиято цел е да допринесе за OBKMS чрез създаването на отворена система за информация за биоразнообразието, основана на знание,  извлечено от научната литература. За да се завърши системата, трябва да бъдат постигнати следните задачи:

\paragraph{Задача 1: Архитектура.} Да се формализира OpenBiodiv като основана на знания система и да се създаде интегрираната \'{и} софтуерна архитектура.

\paragraph{Задача 2: Онтология.} Изучаване на областта на информатиката на биоразнообразието и публикуването на данни за биоразнообразието и разработване на онтология, която позволява интегрирането на данни от различни източници.

\paragraph{Задача 3: Свързани отворени данни.} Създаване на Свързани данни (LOD) въз основа на публикувани таксономични статии, използващи онтологията, определена в Задача 2.

\paragraph{Задача 4: Софтуерна библиотека.} Разработване на инструменти за преобразуване на таксономичните публикации в семантичния модел на онтологията с цел подпомагане на Задача 3.

\paragraph{Задача 5: Методи за работа.}

Разработване на практически методи за работа за непрекъснато преобразуване на таксономичните данни в таксономични публикации и по този начин актуализиране на набора от данни LOD.

\paragraph{Задача 6: Уеб портал.} Създаване на уеб-портал с примерни приложения в допълнение към базата от знания.

\section*{Методология}
\addcontentsline{ТОС}{section}{Методология}

Тази дисертация има ориентация към разработването на методи и инструменти: т.е. нейната цел не е тестването на конкретна научна хипотеза, а по-скоро теоретичният дизайн и практическото прилагане на система за управление на знанията. В този раздел ще очертая ``мета-изборите'', които съм направил - какви парадигми за програмиране и база данни трябва да използват преди фазата на проектиране и изпълнение.

\subsection*{Избор на парадигма на база данни за OpenBiodiv}

\addcontentsline{toc}{subsection}{Избор на парадигма на база данни за OpenBiodiv}

Форумулирахме OpenBiodiv като система, основана на знание, с фокус върху структурирането и взаимовръзката на данни за биоразнообразието. Две от възможните технологии за бази данни, с които системата може да се реализира, са графови семантични бази данни (triple stores), като напр. GraphDB (\cite{ontotext_graphdb_2018}) и чисти графови бази данни (labeled property graphs) Neo4J (\cite{neo4j_developers_neo4j_2012}). Семантичните графови бази данни предлагат много прост модел за данни: всеки факт, съхраняван в такава база данни, е съставен като тройка от подлог \emph{subject}, сказуемо \emph{predicate} и пряко допълнение \emph{object}. Подлозите на тройките са винаги идентификатори на ресурси, докато допълнения могат да бъдат други идентификатори на ресурси или буквални стойности (литерали,  напр. низове, числа и т.н.). Връзките между ресурсите или между ресурсите и литералите са дават от сказуемите (посочени също като идентификатори). Тези връзки понякога се наричат предикати или свойства (\emph{properties}). По този начин може да се визуализира граф, чиито върхове са идентификатори на ресурси или литерали и чиито ребра са свойства.Семантичните графови бази данни имат уникалната черта, че логическият слой също се изразява като тройки, съхранени в базата данни. Този логически слой, известен като онтология (\emph {ontology}), не само отговаря за извличането на изводи от данните (извод), но също така посочва и семантиката на начина, по който трябва да се изрази знанието.

Чистите графове, от друга страна, предлагат по-свободен модел за данни, като позволяват и ребрата на графа от знания да имат етикети или свойства. Например, в граф, чиито върхове са два града $A$ и $B$, които са свързани посредством свойството \emph {свързани с път}, е възможно допълнително да се приложи стойността ``500 км'' на това свойство. По този начин посочваме, че дължината на пътя, свързващ градовете, е 500 км. Имайте предвид, че чистите графове не са по-изразителни от това, което може да се постигне само с тройки. Всъщност, сложни взаимоотношения в обикновен triple store могат да бъдат изразени чрез преобразуването на сложни свойства в нови ресурси, които имат собствени свойства. Този процес е известен като реификация (\emph{reification}). Например, двата града $A$ и $B$ могат да се свържат към друг връх, $R$, посочващ пътя. $R$ ще има три свойства: \emph{start}, \emph{end} и \emph{length}. Стойността (допълнението) на \emph{start} ще бъде $A$, на \emph{end} ще бъде $B$, а \emph{length} ще бъде литералът 500 км или числото 500.

Обобщил съм разликите между чистите графове и семантичните графови базите данни Таблица~\ref {graphdb-vs-neo4k}. След внимателни разсъждения, ние се спряхме на  семантичната графова база данни като избор на технологията на базата данни. Това решение беше информирано за широката наличност на висококачествени онтологии и RDF модели в нашата област (\cite{baskauf_darwin-sw:_2016, peroni_semantic_2014}) и популярността на семантичната мрежа (\cite{berners-lee_semantic_2001}) в общността. Освен това разположението ни в издателска къща ни подтикна към стандартизиран проект, а не към частен случай.

\begin{table}
\caption{Разлики между бази данни върху семантични графове (напр. GraphDB) и чисти графове (напр. Neo4j).}
\begin{tabular}{>{\centering\arraybackslash}m{2.5cm}|>{\centering\arraybackslash}m{4.2cm}|>{\centering\arraybackslash}m{4.2cm}}
Критерий   & Семантичен граф & Чист граф\\
\hline
Семантика & Съхранява се в самата база данни като OWL или RDFS-изрази. Осигурява единно пространство за данни. Изисква експертни онтолози да извличат знания. & Формална семантика обикновено липсва. Бързо внедряване. Унифицирано пространство за данни по-трудно постижимо. \\
\hline
Извод & Осигурен от самата база данни от нейната онтология или формуралиран посредством SPARQL заявки. С общо предназначение, по-бавен. & Външен за базата данни. Трябва да се напише за всяка конкретна задача. Със специално предназначение. По-бърз. \\
\hline
Общност & Има богата и зряла общност от онтолози и инженери по знания. Много онтологии за различните дисциплини. Стремяща се към стандарти. & Моделите се създават ad-hoc от програмисти за дадена задача. Постигането на интер-оперативност на данните изисква усилия и не е от първостепенно значение. Стремяща се към работещи приложения. \\
\hline
\end{tabular}
\label{graphdb-vs-neo4k}
\end{table}

Въпреки това смятам, че чистите графове са по-свободен и по-естествен модел на данни и са напълно подходящи за информатиката за биоразнообразието. По-специално, те осигуряват много по-естествен формализъм за изразяване на взаимоотношенията между таксономичните понятия (обсъдени в глава ~\ref{chapter-ontology}). Също така напоследък семантични бази данни, различни от RDF, като \mbox {WikiData}, стават популярни. Ето защо сичтам, че приложимостта на RDF за OpenBiodiv трябва постоянно да бъде преоценявана.

\subsection*{Избор на източници на информация}
\addcontentsline{toc}{subsection}{Избор на източници на информация}

Съгласно \cite{noauthor_pro-ibiosphere_2014} биоразнообразието и свързаните с биоразнообразието данни имат два различни ``цикъла на живот''. В миналото, след като е било правено наблюдение на жив организъм, то е записвано в тетрадка и след това бележка за наблюдение е публикуванa в научна статия или монография. За да могат данните за биологичното разнообразие да бъдат достъпни за съвременния учен, в днешно време Plazi, както и Библиотеката за наследство на биологичното разнообразие (\cite {miller_taxonomic_2012}), полагат усилия за дигитализиране на публикации, публикувани на хартиен носител (\cite{agosti_why_2007}). За тази цел са разработени няколко специални XML схеми (вж. \cite {penev_xml_2011} за преглед), от които TaxPub (\cite {catapano_taxpub:_2010}) и TaxonX са най-широко използвани (\cite{penev_implementation_2012}). Дигитализирането на публикациите съдържа няколко стъпки. След сканиране и оптично разпознаване на символи (OCR), се извършва text-mining. Тази процедура маркира елементи (semantic markup), които могат да бъдат извлечени и предоставени за бъдещо използване и повторно използване (\cite{miller_integrating_2015}).

В днешно време данните за биоразнообразието и се публикуват в цифров формат като семантично подобрени публикации (EP, \cite{claerbout_electronic_1992, godtsenhoven_van_emerging_2009, shotton_semantic_2009}). Според \cite{claerbout_electronic_1992}, ``ЕП е публикация, която е подобрена с данни от изследвания, допълнителни материали, данни след публикуването и записи в базата данни. Тя има структура, базирана на обекти, с изрични връзки между обектите. Един обект може да бъде (част от) статия, набор от данни, изображение, филм, коментар, модул или връзка към информация в база данни. ''По този начин семантично подобрените публикации са ``родени в Интернет и в семантичната мрежа'' за разлика от техните хартиени предшественици.

Актът за публикуване в цифров, подобрен формат се различава основата си от публикуването в печатен източник. Основната разлика е, че цифрово публикуваният документ може да бъде структуриран в такъв формат, че да е подходящ както за машинна обработка, така и за човешкото око. В сферата на науката за биологичното разнообразие, списания на Pensoft като ZooKeys, PhytoKeys и Biodiversity Data Journal (BDJ) от години предлагат публикуване на ЕП (\cite{penev_semantic_2010}).

Предвид факта, че публикациите на Пенсофт и Plazi покриват голяма част от таксономичната литература както по обем, така и по хронология, и фактът, че публикациите на тези две издателства са достъпни като семантични ЕП, периодичните издания на Пенсофт и Плаци бяха избрани като основни източници на информация.

Освен това, таксономичният гръбнака на GBIF беше включен \cite{gbif_secretariat_gbif_2017} като източник за интеграция на данни. Това се обсъжда допълнително в глава ~ \ref{chapter-lod}.

\subsection*{Избор на методология и среда за програмиране}
\addcontentsline{toc}{subsection} {Избор на методология и среда за програмиране}

През 2016 г., въз основа на резултатите от pro-iBiosphere и на съществуащи разработки в областта на информатиката за биологичното разнообразие, публикувахме план за докторантурата (\cite{senderov_open_2016}). Тази публикация може да се счита за първата спецификация на проекта на OpenBiodiv. Въпреки това, в хода на разработването на системата, нейният дизайн бе променен итеративно чрез обратна връзка от сътрудници от проекта \href {http://big4-project.eu}{BIG4 project}\footnote {Кандидатът Виктор Сендеров е част от Международната мрежа за обучение на ``Мария Склодовска-Кюри'' BIG4: Биосистемата, информатиката и геномиката на големите 4 групи насекоми: обучение утрешните изследователи и предприемачи и различни международни сътрудници.} Разглеждаме тези промени положително и в духа на отворената наука (\emph {open science}) и на \emph{agile разработката на софтуер} (\cite {beck_manifesto_2001}). Този итеративен подход се различава от подхода waterfall, където след фазата на проектиране, спецификациите ``са замразени'' през дълга изпълнителна фаза.

През последните години програмният език R се използва широко в областта на науката на данните data science (\cite{r_core_team_r:_2016}). R има богата библиотека от софтуерни пакети, включваща пакети за обработка на XML (\cite{wickham_xml2:_2018}), за достъп до API (\cite{wickham_httr:_2017}) и се фокусира върху отворената наука (\cite{boettiger_building_2015}). Възможностите на R като функционално-ориентиран и интерпретиран език облекчават подхода за итеративно разработване на софтуер, очертан в предходния параграф. Освен това R се използва широко в общността на информатиката за биологичното разнообразие. Поради тази причина средата на софтуера R е избрана за основна програмна среда.

\subsection*{Отворена наука и Семантична мрежа}
\addcontentsline{toc}{subsection}{Отворена наука и Семантична мрежа}

След като уточнихме избора на базата-данни и на езика на програмиране,  бих искал да обсъдя някои методологии, които правят резултатите ни отворени и възпроизводими.

Вярвам, че OpenBiodiv трябва да бъде разгледан от гледна точка на \emph{Open Science}. Съгласно \cite {kraker_case_2011} и \cite{noauthor_was_nodate}, шестте принципа на Отворената наука са: отворена методология, отворен код, отворени данни, отворен достъп, отворени рецензии и отворени образователни ресурси. Моето убеждение е, че целта на Отворената наука е да осигури достъп до всички изследователски продукти: данни, открития, хипотези, код и т.н. Това отваряне гарантира, че научният продукт може да бъде възпроизводим и проверим от други учени (\cite{mietchen_transformative_2014}). Съществува голям интерес към разработването на процеси и инструменти, които дават възможност за възпроизводимост и проверка. Тези проблеми са разгледани напр. в специален брой в Nature, посветена на възпроизводими изследвания (\cite {noauthor_challenges_2010}). Поради това изходният код, данните и публикациите на OpenBiodiv ще бъдат публикувани открито.

Освен това OpenBiodiv трябва да се разглежда като неразделна част от семантичната мрежа (\cite{berners-lee_semantic_2001}). Семантичната мрежа е визия за бъдещето на мрежата, където са свързани не само документи, но и данни.

\section*{Структура на дисертацията}
\addcontentsline{toc}{section}{Структура на дисертацията}

Дотук в настоящото въведение беше дадена мотивировка за съществуването на системата OpenBiodiv, както и обобщение на нейните цели и задачи.

В глава~\ref{chapter-openbiodiv} ще бъде представена формалната спецификацията и дизайна на желаната система, както и на нейната архитектура; тази глава формира Задача 1. Следващите глави обсъждат изпълнението на OpenBiodiv. Глава~\ref{chapter-ontology} предлага формална концептуализация на областта на публикуването на данни за биологичното разнообразие. Въвежда се централният резултат на дисертацията - онтологията OpenBiodiv (OpenBiodiv-O), Задача 2. Глава~\ref{chapter-lod} описва Отворените данни, които са генерирани въз основа на OpenBiodiv-O и формира Задача 3. Глава~\ref{chapter-rdf4r} подробно описва софтуерния пакет RDF4R (R пакет за работа с RDF), който бе използван за създаване на Linked Open Data (OpenBiodiv-LOD) и формира Задача 4. В глава~\ref{chapter-case-study} се обсъждат два казуса за внасяне на данни в OpenBiodiv от важни международни хранилища, Задача 5. Глава~\ref{chapter-webportal} обсъжда уебсайта, който се подготвя да служи на на OpenBiodiv-LOD и приложенията му (Задача 6).
В Заключение ще обясня, как са публикувани резултатите, и ще обобщя основните резултати.


%----------------------------------------------------------------------------------------



