\chapter{Резюме на глава 4: Библиотека R за работа с RDF}
\label{chapter-rdf4r}

RDF4R (\cl{rdf4r}) е R пакет за работа с RDF. Тя е разработена като част от проекта OpenBiodiv, но е напълно свободна от специфичен за OpenBiodiv код и може да се използва за общи цели, изискващи инструменти за работа с RDF данни в средата за програмиране R (\cite {r_core_team_r: _2016}).

В тази глава разяснявам инсталирането и функционалността на пакета.

\section{Спецификация}

Основните фукнции на пакета са:

\subsection{Връзка със семантична база данни}

\subsection{Функции за преобразуване на SPARQL заявки в R функции}

\subsection{Работа с литерали и идентификатори}

\subsection{Работа с префикси}

\subsection{Създаване и сериализация на RDF}

\subsection{Основна терминология на семантични понятия}

\section{Употреба}

Употребата на пакета е обяснена в тази секция с помощта на примери.

\section{Дискусия}

\subsection{Сходни пакети}

Дискутираме приликите и разликите на RDF4R с подобни пакети като \cl{rdflib}.

\subsection{Елементи на функционалното програмиране (FP)}

В тази подсекция обсъждаме как са използвани модели от функционалното програмиране за създаване на RDF4R.

\subsection{Елементи на обектно-ориентирано програмиране (OOP)}

В тази подсекция обсъждаме как се използват модели от обектно-ориентирано програмиране за създаване на RDF4R.
