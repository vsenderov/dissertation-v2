% Chapter Template

\chapter{Web portal} % Main chapter title
\label{chapter-webportal}

%----------------------------------------------------------------------------------------
%	SECTION 1
%----------------------------------------------------------------------------------------

Under \href{http://openbiodiv.net}{\url{openbiodiv.net}} one can reach the main portal giving access to OpenBiodiv resources. This portal was developed by Pensoft to support OpenBiodiv. \href{http://openbiodiv.net}{OpenBiodiv.net} presents two visual elements to the user: the search bar and list of application icons in the bottom. Furthermore, under \href{http://graph.openbiodiv.net}{\url{graph.openbiodiv.net}} (also accessible from the icon SPARQL endpoint) one can reach the OpenBiodiv workbench, a feature of GraphDB that gives web access to the SPARQL endpoint.

These User Interface (UI) features are designed to facilitate the three user types of the system that we envisage:

\begin{enumerate}
    \item Basic level.
    \item Specialist level.
    \item Power user.
\end{enumerate}

The basic level of interaction is for users who want a quick look into the system's database; they can be beginners without knowledge of the Semantic Web or of taxonomic, or advanced users with little time or a very basic query. An example of such a user will simply look for an entity (e.g. taxonomic name, person) and would like to retrieve some information about it.


\section{Functionality of the portal}

