\chapter{Conclusion}
\label{chapter:summary}

\section{List of publications}

\subsection{Publications in international scientific journals}

A list of publications related to the dissertation follows, which are all in international scientific journals. [1] is an early version of the Introduction as well Chapter~\ref{chapter-openbiodiv} and contains work towards Objective 1 (Architecture). [2, 3, 5, 6, 7] are not a part of the text of the dissertation but can be considered work towards Objective 5 (Workflows). [4] is also a publication towards Objective 5 (Workflows) and a version of it served as the bases for Chapter~\ref{chapter-case-study}. [7] is published in the peer-reviewed journal ZooKeys with impact factor 1.031 (early 2018). [8] is the most important publication under this dissertation and was published in the high-impact Journal of Biomedical Semantics with impact factor 2.413 (early 2018). [8] makes up the content of Chapter~\ref{chapter-ontology} and is the main body of work fulfilling Objective 2 (Ontology). It was a featured article on the home-page of JBS 

\begin{enumerate}
\item \fullcite{senderov_open_2016}
\item \fullcite{sarah_faulwetter_emodnet_2016}
\item \fullcite{cardoso_species_2016}
\item \fullcite{senderov_online_2016}
\item \fullcite{penev_strategies_2017}
\item \fullcite{penev_arpha-biodiv:_2017}
\item \fullcite{arriaga-varela_review_2017}
\item \fullcite{senderov_openbiodiv-o:_2018}
\end{enumerate}

\section{Aprobaciya na rezultatite}

\section{Main scientific and applied advances}

In the course of the investigative effort, we have made several significant contributions. The two equally important key contributions of the thesis are the creation of an ontology, OpenBiodiv-O (see Chapter~\ref{chapter-ontology}), enabling the linking of biodiversity knowledge on the basis of scholarly publications, and a linked open dataset, OpenBiodiv LOD (see Chapter~\ref{chapter-lod}) consisting of a transformation to Resource Description Framework (RDF) and integration in a single store of information from three major repositories of biodiversity data: the journal databases of Pensoft Publishers and Plazi, and GBIF's taxonomic backbone.

OpenBiodiv-O, serves as the basis of OpenBiodiv-LOD. By developing an ontology focusing on biological taxonomy, our intent is to provide an ontology that fills in the gaps between ontologies for biodiversity resources such as Darwin-SW and semantic publishing ontologies such as the ontologies comprising the SPAR Ontologies. Moreover, we take the view that it is advantageous to model the taxonomic process itself rather than any particular state of knowledge.

After the populating the ontology, we have developed a web-site (see Chapter~\ref{chapter-openbiodiv}, \href{http://openbiodiv.net}{<http://openbiodiv.net>}, containing a semantic search engine and apps running on top of OpenBiodiv LOD.

Furthermore, we have developed two automated workflows---automatic data paper generation and automatic occurrence record import, see Chapter~\ref{chapter-case-study}---between external repositories and the Pensoft database of biodiversity data.

Last but not least is the creation of a framework for transforming XML and CSV files to RDF in the R programming language (RDF4R, see Chapter~\ref{chapter-rdf4r}).

\section{Deklaratziya za originalnost}