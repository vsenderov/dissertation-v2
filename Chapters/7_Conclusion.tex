\chapter*{Conclusion}
\addcontentsline{toc}{chapter}{Conclusion}
\label{chapter:summary}

\section*{Results}
\addcontentsline{toc}{section}{Results}

We believe that the presented scientific work fulfills the stated objective and tasks. In this section, we will summarize the results achieved and how they relate to the fulfillment of the stated objectives and tasks. In the course of the study all six tasks were achieved and the results were published in prestigious international journals and presented at major conferences in Bulgaria and abroad. The most important results of the thesis are summarized as follows:

\paragraph{Result 1.} The cenral result of the thesis is the creation of a domain conceptualization of biodiversity publishing and a formal ontology OpenBiodiv-O enabling the linking of biodiversity knowledge on the basis of scholarly publications. OpenBiodiv-O serves as the basis of the Linked Open Data OpenBiodiv-LOD. By developing an ontology focusing on biological taxonomy, we provided an ontology that fills in the gaps between ontologies for biodiversity resources such as Darwin-SW and semantic publishing ontologies such as the ontologies comprising the SPAR Ontologies. Moreover, we take the view that it is advantageous to model the taxonomic process itself rather than any particular state of knowledge. This result has been described in Chapter~\ref{chapter-ontology} and in \cite{senderov_openbiodiv-o:_2018} and fulfills Objective 1. The source code of the ontology is available under \href{https://github.com/pensoft/openbiodiv-o}{github.com/pensoft/openbiodiv-o}. At this stage, the coverage of the ontology of the different types of resources is sufficient to be the basis for creating the LOD. In this sense, it is completed. On the other hand, adding classes and properties for new types of biodiversity data is possible and desirable.

\paragraph{Result 2.} The second result of the thesis is the creation of the software architecture of the OpenBiodiv system outlined in Chapter~\ref{chapter-openbiodiv} and \cite{senderov_open_2016}. This result fulfills Objective 2.

\paragraph{Result 3.} The third result of the thesis has been the creation of a Linked Open Dataset, OpenBiodiv-LOD, consisting of a transformation to RDF-triples and integration in a single store of information from three major repositories of biodiversity data: the XML sources of biological journals published by Pensoft Publishers, the XML sources of treatments freed by Plazi, and a CSV dump of GBIF's taxonomic backbone. OpenBiodiv-LOD is available under \href{http://graph.openbiodiv.net}{\url{graph.openbiodiv.net}} and has been described in Chapter~\ref{chapter-lod}. This result fulfills Objective 3. The LOD, similar to the ontology, are already a solid resource for biologists, as they include information from most articles published by Pensoft and Plazi and count over 600 million triplets. Like the ontology, they should be expanded.

\paragraph{Result 4.} In order to  create the Linked Open Data, a software package for the R programming environment, RDF4R, was developed. RDF4R enables the manipulation of RDF data within R and facilities the transformation of scientific publications from a semi-structured XML format to structured semantic RDF. This result has been discussed in Chapter~\ref{chapter-rdf4r} and fulfills Objective 4. The package is available online as free software under \href{http://github.com/pensoft/rdf4r}{\url{github.com/pensoft/rdf4r}}. Furthermore, additional source code (unoptimized) describing XML schemas of Pensoft and Plazi and working in tandem with RDF4R to convert XML to RDF can be found under \href{http://github.com/pensoft/ropenbio}{\url{github.com/pensoft/ropenbio}}. Since the package was successfully used to create an LOD, it can be considered complete. Like any software package, however, it should be maintained and developed. Details on the directions for its development are listed in the next section.

\paragraph{Result 5.} The mechanisms to convert semi-structured XML into RDF-triples are complemented by workflows enabling the enrichment of the XML sources of Pensoft journals by data automatically imported from the major international biodiversity data repositories: BOLD, GBIF, iDigBio, as well as PlutoF. Furthermore, it is now possible, thanks to this dissertation effort to automatically create manuscripts from metadata encoded in the Ecological Metadata Language (EML). The discussion of these automated workflows---automatic data paper generation and automatic occurrence record import---is carried out in Chapter~\ref{chapter-case-study}. It fulfills Objective 5.

\paragraph{Result 6.} To complement the creation of OpenBiodiv-LOD, we have developed a website running on top of the knowledge graph \href{http://openbiodiv.net}{openbiodiv.net}, containing a semantic search engine and apps. The website is discussed in Chapter~\ref{chapter-webportal} and fulfills Objective 6. The website is still in beta. The functionality that works great is the semantic search engine. For some basic data types there are templates for visualization. However, the site can not be considered complete and most users use the SPARQL search language.

\section * {Summary and outlook}
\addcontentsline {toc} {section} {Summarizing discussion of the results and future directions}

An important conclusion that can be drawn from the work is that it is possible to use a semantic graph for the integration of a large volume of data on biodiversity. We were unexpectedly given the opportunity to illustrate the power of the knowledge graph by analyzing the damage from the tragic fire at the Museu Nacional in Rio de Janeiro. In addition, we have illustrated that it is possible to write relatively simple logical conclusions to check the validity of a taxonomic name.

Due to the large amount of data, we found that although the use of a semantic graph was possible, some of the initially chosen technologies proved to be inapplicable or difficult to apply. We have observed (see Chapter~\ref {chapter-lod}) that the practical application of the full logical OWL model is difficult due to performance problems. Instead in the end, we utilized RDFS that is less powerful but faster. Another observation of ours is that although the R programming environment has given us some advantages in rapidly creating the prototype of the system, by increasing the complexity of the program code needed in the real-life system to cover all private cases, a language with dynamic types such as R creates headaches in debugging. At the same time, we were impressed by the powerful functional programming toolkit R provided.

A big difficulty was the disambiguation of resources such as author names or taxonomic names. In the functional design of the RDF4R package we have put modules that allow us to insert a list of functions/rules for disambiguation when searching for an identifier for a given resource. However, we had only limited success with the rule-based disambiguation and for this reason in the production system it was discontinued at the moment.

Considering these and other ``lessons,'' the future development of the OpenBiodiv project can be outlined in the following not necessarily comprehensive way:

\begin{enumerate}
    \item As an immediate goal, to expand the LOD and ontology with new data types and new data sources using the existing framework. Such data are e.g. genomic data, occurrence data, (bio-)geographic data, visual data, descriptive data, etc.
    \item Look for even closer integration with other existing biodiversity data repositories than GBIF. For example, BioImages, iNaturalist, BOLD, and so on.
    \item As a longer-term task to study the transition from a semantic graph to a technology where the inference engine is separated from the data base layer as WikiData or Neo4j. In addition to increased performance, this will give extra flexibility to the project, such as allowing the use of non-RDF-based inference engines such as Euler.
    \item Continue developing system software with an even wider application of functional programming and porting it into a functional language like, for example, Haskell or O'CAML.
    \item To investigate the problem of disambiguation and related problems for named entity recognition of interesting resources from biodiversity, as well various image recognition tasks, from the point of view of machine learning.
    \item Expanding the website with more templates and new applications.
\end{enumerate}

\section*{Key scientific and applied contributions}
\addcontentsline{toc}{section}{Key scientific and applied contributions}

The results discussed in the previous two sections determine the following scientific and applied contributions:

\begin{enumerate}
    \item Scientific contribution: creating an ontology and a formal model of the field of biodiversity knowledge publication.
    \item Applied scientific contribution: analyzing information sources and Creating OpenBiodiv-LOD.
    \item Applied scientific Contribution: the implementation of OpenBiodiv software modules.
\end{enumerate}

Our ontology fills the unique niche between bibliographic ontologies such as SPAR and ontologies for biodiversity such as Darwin-SW and as such is undoubtedly of great scientific interest to the biodiversity informatics community. The work has a serious scientific and applied character by providing both a Linked Open Dataset on top of the ontology and software for its users and system developers.

\section*{List of publications}
\addcontentsline{toc}{section}{List of publications}

\subsection*{Publications in international scientific journals}
\addcontentsline{toc}{subsection}{Publications in international scientific journals}

\begingroup
\newcounter{count}
\setcounter{count}{99}
\defcounter{maxnames}{\value{count}}%

\begin{enumerate}
\item \longfullcite{senderov_open_2016}. Unique citations by \cite{franz_increase_2018}, \cite{ordynets_aphyllophoroid_2017} and \cite{burt_origin_2017}.
\item \longfullcite{sarah_faulwetter_emodnet_2016}. Unique citation by \cite{pyron_21st_2018}.
\item \longfullcite{cardoso_species_2016}. Indexed in WoS SCOPUS, as well as SJR $0.465$. Unique citations by \cite{bachman_quantifying_2018}, \cite{lin_draft_2017}, \cite{li_genomic_2017}, \cite{milano_conservazione_2017}.
\item \longfullcite{senderov_online_2016}. Unique citations by \cite{ordynets_aphyllophoroid_2017}.
\item \longfullcite{penev_strategies_2017} Unique citations by \cite{tennant_multi-disciplinary_2017}, \cite{marwick_standard_2017},  \cite{kissling_towards_2018},  \cite{mathieu_egrowth:_2018}, \cite{__2018}, \cite{__2017}, \cite{filippova_biodiversity_2017},  \cite{__2017-1}.
\item \longfullcite{penev_arpha-biodiv:_2017} 
\item \longfullcite{arriaga-varela_review_2017}. Indexed in WoS IF $1.079$, Q3 SCOPUS, SJR $0.533$.
\item \longfullcite{senderov_openbiodiv-o:_2018} Indexed in WoS IF $1.6$, Q3 SCOPUS, SJR $0.952$. Unique citations by \cite{michel_modelling_2018}, \cite{page_ozymandias:_2018}, \cite{page_liberating_2018}.
\end{enumerate}
\endgroup

A list of publications related to the dissertation follows. The listed articles have been published without exception in four international scientific journals: five articles in Research Ideas and Outcomes, one article in ZooKeys (WoS IF $1.079$, Q3 SCOPUS, SJR $0.533$), one article in Biodiversity Data Journal (WoS SCOPUS, SJR 0.465) and one article in Journal of Biomedical Semantics (WoS IF $1.6$, Q3 SCOPUS, SJR $0.952$). The total number of citations that have been accumulated for the candidate excluding self-citations (cross-citations) is at least 20. The citing articles are given in the list above. The total number of citations that have been accumulated including cross-citations and citations of work outside of the scope of the dissertation is at least 48 (Google Scholar).

[1] is an early version of the Introduction as well Chapter~\ref{chapter-openbiodiv} and contains work towards Objective 2 (Architecture). The text of publications [2, 3, 5, 6, 7] are not a part of the text of the dissertation one-to-one but contain work towards Objective 5 (Workflows). The ideas presented in these publications have to large degree been incorporated in Chapter~\ref{chapter-case-study} whose backbone is formed by [4]; thus Objective 5 (Workflows) is achieved. [7] is published in the peer-reviewed journal ZooKeys with impact factor 1.031 (early 2018). [8] is the most important publication under this dissertation and was published in the high-impact Journal of Biomedical Semantics with impact factor 2.413 (early 2018). [8] makes up the content of Chapter~\ref{chapter-ontology} and is the main body of work fulfilling Objective 1 (Ontology). It was a featured article on the home-page of JBS (Fig.~\ref{fig:jbs-featured}). Chapter~\ref{chapter-lod} and Chapter~\ref{chapter-rdf4r} that form Objectives 3, 4, respectively are currently being prepared as manuscripts in international journals. Furthermore, the software library RDF4R described in Chapter~\ref{chapter-rdf4r} is being submitted to the open source repository rOpenSci\footnote{``We build software with a community of users and developers, and educate scientists about transparent research practices.'' \url{https://ropensci.org/}}.

\begin{figure}
\centering
\includegraphics[width=\textwidth]{Figures/JBS-featured.jpg}
\decoRule
\caption{The OpenBiodiv-O article is featured on the main webpage of the Journal of Biomedical Semantics..}
\label{fig:jbs-featured}
\end{figure}

\section*{Апробация на резултатите}
\addcontentsline{toc}{section}{Апробация на резултатите}

\subsection*{Доклади пред научен семинар на ПНЗ}
\addcontentsline{toc}{subsection}{Доклади пред научен семинар на ПНЗ}

\begin{enumerate}
    \item Доклад пред научен семинар на ИБЕИ на БАН на 26.10.2015 г. (“Публикуване, визуализация и разпространение на първични и геномни данни за биологичното разнообразие на основата на открита система за управление на информацията”).
    \item Доклад пред научен семинар в ИИКТ на БАН на 31.03.2016 г. (Open Biodiversity Knowledge Management System)
    \item Dоклад пред научен семинар на ИИКТ на БАН за 23.03.2018 г. (OpenBiodiv: a knowledge-based system of biodiversity information)
\end{enumerate}

\subsection*{Доклади пред научно мероприятие в чужбина или пред международно научно мероприятие у нас}
\addcontentsline{toc}{subsection}{Доклади пред научно мероприятие в чужбина или пред международно научно мероприятие у нас}

\begin{enumerate}
    \item Доклад пред международния симпозиум EU BON в София на 23.03.2016 г. (The Data Publishing Toolkit at EU BON: Automated creation of data papers, data and text integrated publishing via the ARPHA Publishing Platform.)
    \item Доклад по време на работната среща на BIG4 в Хавраники, Чехия на 03.06.2016 г. (Project Progress Report (OBKMS))
    \item Доклад по време на работната среща на BIG4 в Хавраники, Чехия на 03.06.2016 г. (Modern Methods of Systematic Research and the BOLD Algorithm)
    \item Уеб-базиран доклад (уебинар) пред международна аудитория в рамките на семинар на iDigBio на 16.07.2016 г. (Online direct import of specimen records from iDigBio instrastructure into taxonomic manuscripts)
    \item Доклад по време на работната среща на BIG4 в Копенхаген на 14.10.2016 г. (Midterm Progress Report)
    \item Доклад на международия симпозиум TDWG 2016 в Санта Клара де Сан Карлос от 5. до 9.12.2016 г. (Streamlining the Flow of Taxon Occurrence Data Between a Manuscript and Biological Databases)
    \item Доклад на международия симпозиум TDWG 2016 в Санта Клара де Сан Карлос от 5. до 9.12.2016 г. (The Open Biodiversity Knowledge Management System: A Semantic Suite Running on top of the Biodiversity Knowledge Graph)
    \item Доклад на международия симпозиум TDWG 2016 в Санта Клара де Сан Карлос от 5. до 9.12.2016 г. (Demonstrating the Prototype of the Open Biodiversity Knowledge Management System)
    \item Доклад на международия симпозиум TDWG 2016 в Санта Клара де Сан Карлос от 5. до 9.12.2016 г. (Creation of Data Paper Manuscripts from Ecological Metadata Language (EML))
    \item Уеб-базиран доклад пред междуродния семинар на работната група по семантични технология към Университета Вандербилт (Тенеси, САЩ) на 20.02.2017 г. (Open Biodiversity Knowledge Management System)
    \item Доклад на европейската конференция на биосистематиците, BioSyst.eu 2016 на 15.08.2017 г. (The OpenBiodiv Knowledge System: The Future of Access to Biodiversity Knowledge)
    \item Доклад на международия симпозиум TDWG 2017 в Отава, Канада от 1. до 6.10.2017 г. (OpenBiodiv Computer Demo: an Implementation of a Semantic System Running on top of the Biodiversity Knowledge Graph)
    \item Доклад на международия симпозиум TDWG 2017 в Отава, Канада от 1. до 6.10.2017 г. (OpenBiodiv: an Implementaion of a Semantic System Running on top of the Biodiversity Knowledge Graph)
    \item Постер на международия симпозиум TDWG 2017 в Отава, Канада от 1. до 6.10.2017 г. (OpenBiodiv: an Implementaion of a Semantic System Running on top of the Biodiversity Knowledge Graph)
    \item Доклад по време на работната среща на BIG4 в Ла Палма, Испания от 30. окт. до 3 ноем. 2017 г. (Midterm Progress Report)
    \item Доклад пред научен семинар на групата по биоинформатика (група Ронкуист) в Кралския природо-научен музей в Стокхолм на 29.11.2017 г.
\end{enumerate}

\section*{Main scientific and applied contributions}
\addcontentsline{toc}{section}{Main scientific and applied contributions}

In the course of the investigative effort, all six objectives have been achieved and the results have been published in international journals and have been presented at major conferences in Bulgaria and abroad. The most important contributions of the thesis are summarized as follows:

\section*{Декларация за оригиналност}
\addcontentsline{toc}{section}{Декларация за оригиналност}

Декларирам, че настоящата дисертация съдържа оригинални резултати, получени 
при проведени от мен научни изследвания, с подкрепата и съдействието на научния ми ръководител проф. Любомир Пенев и Издаделство Пенсофт, както и научния ми консултант доц. Кирил Симов и ИИКТ.  Резултатите,  които  са  получени,  описани  и/или  публикувани  от  други учени, са надлежно и подробно цитирани в библиографията.

Настоящата дисертация не е прилагана за придобиване на научна степен в друго 
висше училище, университет или научен институт.

Виктор Сендеров

%----------------------------------------------------------------------------------------
%	ACKNOWLEDGEMENTS
%----------------------------------------------------------------------------------------

\begin{acknowledgements}
\addchaptertocentry{\acknowledgementname} % Add the acknowledgements to the table of contents

This research has been financed through the European Union’s Horizon 2020 research and innovation program under the Marie Sklodowska-Curie grant agreement No. 642241. My deep gratitude goes to the European Commission for enabling this wonderful opportunity!

\vspace{5mm}

I thank Prof. Lyubomir Penev and Prof. Kiril Simov for the valuable supervision. I also thank the staff and developers at Pensoft Publishers for the support in creating the platform and its popularization; in particular Prof. Pavel Stoev, Teodor Georgiev, Georgi Zhelezov, Iliyana Kuzmova, and Iva Kostadinova. Furthermore, I thank Pensoft's graphic designer, Slavena Peneva, for the help with creating the illustrations for this thesis and in presentations. Last but not least, Margarita Grudova and Elisaveta Taseva for providing valuable administrative support during the elaboration of the thesis.

\vspace{5mm}

I thank my colleagues from the Bulgarian Academy of Sciences (Institutes for Information and Communication Technologies and for Biodiviversity and Ecosystems Research) for their friendship and advice; in particular Prof. Galya Angelova, Prof. Boyko Georgiev, and Prof. Snejana Grozeva.

\vspace{5mm}

I thank my colleagues at the BIG4 training network for the feedback, friendship, and support. In particular Prof. Alexey Solovdnikov, but there are too many more names to mention.

\vspace{5mm}

I thank my international collaborators for their ideas, reviews, and collaboration on papers. In particular Prof. Nico Franz (Arizona State University), Dr. Daniel Mietchen (National Institutes of Health), Dr.  Éamonn Ó Tuama (formerly at GBIF), and Prof. Bob Morris (emeritus UMASS).

\vspace{5mm}

I also thank everyone at Plazi for the co-ownership of the vision of the project; in particular, Dr. Donat Agosti, Terry Catapano, and Dr. Guido Sautter.

\vspace{5mm}

Last but not least, I would like to acknowledge Ontotext for building the GraphDB database and providing excellent support.



\end{acknowledgements}