\chapter{Резюме на глава 1: Архитектура на OpenBiodiv}
\label{chapter-openbiodiv}

В тази глава предоставяме архитектурата, т.е. спецификацията и дизайна на OpenBiodiv. Въвеждаме компонентите на OpenBiodiv, които ще бъдат разгледани подробно в следващите глави. Описваме как взаимодействат тези компоненти, за да се формира базираната на знанието система OpenBiodiv.

\section{Какво е OpenBiodiv?}

Разбирането за OpenBiodiv като система базирана на знанието може да се обобщи по следния начин: OpenBiodiv е база данни от взаимосвързана информация за биологичното разнообразие, заедно с логика и приложения, позволяващи на потребителите не само да се допитват до данните, но и да откриват допълнителни факти, свързани с данните. Основните източници на информация в OpenBiodiv са списанията на академичния издател Пенсофт, таксономичната информация от Плаци и таксономичният гръбнак на Global Information Biodiversity Facility (GBIF).

Изследователският проблем на архитектурата на OpenBiodiv може да се формулира като проектиране на семантична графична база данни на основата на RDF. Тя да е с отворен достъп и да включва информация, предоставяна от Пенсофт, Плаци и GBIF, и да позволява на потребителите на системата да задават сложни заявки.

OpenBiodiv се състои от (1) семантична графична база данни, (2) програмен код, осигуряващ функционирането на базата и (3) динамична уеб страница (front-end), улесняваща достъпа до основната база от знания (Фиг.~\ref{fig:openbiodiv-components}). OpenBiodiv позволява динамичното вмъкване на данни от хранилища за данни за биологичното разнообразие в текста на статия в Biodiversity Data Journal или друго списание, използващи фреймуорка на Пенсофт за писане на статии, ARPHA-BioDiv (\cite{penev_arpha-biodiv:_2017}). Като втора стъпка, от тези списания се извлича знание, като се възползваме от XML-схемата на тези списания (TaxPub). Списания, които са извлечени, включват ZooKeys, Biodiversity Data Journal (BDJ), PhytoKeys, MycoKeys и т.н.\footnote{Списанията могат да бъдат достъпвани под \url{https://pensoft.net/browse_journals}}. В същото време знание под формата на факти (трипъли) се извлича от Plazi TreatmentBank, архив на литература за биологичното разнообразие, съдържащ над 200 хиляди таксономични дискусии\footnote{Таксономичната дискусия е специален раздел в биологична публикация, която описва и дискутира вид или по-висок таксон. TreatmentBank е достъпен под \url{https:// //plazi.org/resources/treatmentbank/}} и актуализиран всеки ден. Не на последно място, тези факти са взаимосвързани чрез таксономичния гръбнак на GBIF (\cite{gbif_secretariat_gbif_2017}). След това извлеченото знание се съхранява в нашата семантична база данни  (Фиг.~\ref{fig:openbiodiv-sources}).

\begin{figure}
\centering
\includegraphics[width=\textwidth]{Figures/components-openbiodiv}
\decoRule
\caption[OpenBiodiv Components]{Компоненти на OpenBiodiv.}
\label{fig:openbiodiv-components}
\end{figure}

\begin{figure}
\centering
\includegraphics[width=\textwidth]{Figures/openbiodiv-sources}
\decoRule
\caption[OpenBiodiv Components]{Поток на информация в пространството за данни за биологичното разнообразие. Пунктираните линии са компоненти, които все още не са създадени.}
\label{fig:openbiodiv-sources}
\end{figure}

\section{Семантична база данни}

Основен резултат от усилията на OpenBiodiv е създаването на семантична база данни, базирана на знания, извлечени от архивите на Пенсофт и Плаци и таксономичния гръбнак на GBIF и достъпни под \url{http://graph.openbiodiv.net/}. Следва обсъждане на компонентите на базата данни.

\subsection{Онтология OpenBiodiv-O}

Централният резултат от усилията по OpenBiodiv е създаването на формален модел на областта за публикуването на знание за биоразнообразието. Този формален модел е онтологията OpenBiodiv-O (\cite{senderov_openbiodiv_2017}). Изходният код на онтологията и придружаващата документация могат да бъдат достъпвани под \url{https://github.com/pensoft/openbiodiv-o}. Детайлна дискусия е представена в глава~\ref{chapter-ontology}.

\subsection{Свързани отворени данни OpenBiodiv-LOD }

Използвайки OpenBiodiv-O и инфраструктурата, описана по-нататък в тази глава, са създадени свързани отворени данни, включващи приблизително 200 хиляди записа от Плаци, пет хиляди статии от Пенсофт, както и таксономичия гръбнак на GBIF (над милион биологични имена). Данните са достъпни онлайн чрез работния инструмент на семантичната база данни \url{http://graph.openbiodiv.net}. Разясняват се подробно в глава~\ref {chapter-lod}.

\section{Backend}

За да се попълва семантичната база данни, е необходимо да се създаде инфраструктура, която преобразува необработени данни (текст, изображения, таблици с данни и т.н.) в структуриран семантичен формат. OpenBiodiv предоставя инфраструктура за трансформиране на научни публикации за биоразнообразието в твърдения под формата на RDF с помощта на инструментите, описани в този раздел.

\subsection{RDF4R: R пакет за работа с RDF}

Едно от по-големите технически предизвикателства за OpenBiodiv е трансформирането на информация за биологичното разнообразие (напр. таксономични имена, метаданни, фигури и т.н.), съхранявани като полу-структуриран XML в напълно структурирани семантични знания под формата на RDF. За да се реши това предизвикателство, е разработен R пакет, който позволява създаването, манипулирането и записа в семантична база данни на създадения RDF. Този пакет е достъпен под лиценз с отворен код на GitHub под \url{https://github.com/vsenderov/rdf4r}. Описваме пакета в глава~\ref{chapter-rdf4r}.

\subsection{Базисен програмен код и ROpenBio}

В комбинация с пакета RDF4R, програмният код съдържа още един R пакет, \cl{ropenbio} и базисен програмен код от скриптове и документация, необходими за стартиране на базата данни. \cl{ropenbio} използва пакета RDF4R за преобразуване на полу-структуриран XML в RDF. Той съдържа преобразуванията, необходими за тази реализация. Той е достъпен под \url{https://github.com/pensoft/ropenbio}. Базисният софтуерен код координира извикването на \cl{ropenbio}, съдържа скриптове за автоматично импортиране на нови ресурси и други подробности. Той е достъпен под \url{https://github.com/pensoft/openbiodiv}. Генерирането на OpenBiodiv-LOD с помощта на тези пакети е обсъдено в глава~\ref{chapter-lod}.

\subsection{Работен процес за преобразуване на екологични метаданни в ръкопис}

Език за екологични метаданни (EML) е популярен формат за описване на екологични данни (\cite{michener_nongeospatial_1997}). Хранилища на данни за биоразнообразието, като GBIF и DataOne, използват този формат за метаданните, които съхраняват. Автоматичното преобразуване на EML файл в data paper ръкопис от Biodiversity Data Journal\footnote {Data paper (\cite{chavan_data_2011}) е научна статия, обсъждаща научни данни.} е възможно с помощта на системата OpenBiodiv (\cite{senderov_online_2016}). Този работен процес е описан подробно в глава~\ref{chapter-case-study}\footnote{За работа в интерактивен режим, отидете на \url{https://arpha.pensoft.net}, влезте в системата (регистрацията е безплатна), изберете ``Start a new manuscript'', превъртете до ``Import manuscript'' и следвайте необходимите стъпки, за да качите EML файл и да го използвате като шаблон за вашия нов ръкопис.}.

\subsection{Работен процес за импортиране на данни за наблюдения на видове в ръкопис}

Един от важните видове данни за биологичното разнообразие са данни за наблюдения на организми, occurrence data. Това са данни, които документират наличието на правилно таксономично идентифициран организъм на дадено място и време. Такива данни се съхраняват в международни хранилища като BOLD, GBIF, PlutoF и iDigBio. За да се улесни публикуването на такъв тип данни е разработен работен процес за импортиране на такива записи от тези бази данни в таксономична статия (taxonomic paper) в списанието Biodiversity Data Journal (\cite{senderov_online_2016}). Този работен процес е описан подробно в глава~\ref{chapter-case-study}\footnote{За да отворите интерактивно работния процес, отидете на \url{https://arpha.pensoft.net}, влезте в системата (регистрацията е безплатна), изберете ``Start a new manuscript'', изберете ``Biodiveristy Data Journal'', ``Taxonomic paper''  и ``Create a new manuscript''. След това в новия ръкопис щракнете с мишката върху раздела ``Taxon treatments'', като кликнете върху знака $+$ до него, и укажете биологичната класификация на новия запис (напр. Animalia), и най-накрая щракнете върху ``Save'' и ще ви бъде представен избор на подраздели. Кликнете върху секцията ``Materials'' от-вляво, за да видите инструмента на работния процес за вмъкване на материли. Погледнете в долната част на диалоговия прозорец, където можете да поставите множество идентификационни номера. Това е частта, в която избирате външни идентификатори на ресурси, които да бъдат импортирани във вашата статия.}.

\section{Интерфейс}

В допълнение към предоставения endpoint за база данни с възможност за търсене, се разработва уебсайт, позволяващ семантично търсене и капсулиращ специфични задачи, пакетирани като приложения (\url{http://openbiodiv.net}). Бета версията вече е в действие. Фиг.~\ref{fig:website}. Ограничена дискусия е представена в глава~\ref{chapter-webportal}.

\section{Системна администрация}

Системата е разположена на виртуална машина на Debian GNU+Linux. GraphDB работи с heap файл от 20 GB и с набор от правила RDFS-Plus Optimized. Това се налага поради факта, че достигнахме препятствия, когато използвахме OWL извод. Обсъдени в глава~\ref {chapter-lod}. Непрекъснатата работа се осигурява от автоматичното изпълнение на скриптове от директорията \cl{run} на базисния код на OpenBiodiv.

\begin{figure}
\centering
\includegraphics[width=\textwidth]{Figures/openbiodiv-webpage}
\decoRule
\caption[OpenBiodiv Website]{Бета версия на потребителския интерфейс.}
\label{fig:website}
\end{figure}

\section{Дискусия}

Проектирането на системата започна във втората половина на 2015 г., когато бяха разгледани различни алтернативи за база данни (Neo4J, GraphDB, WikiBase) и различни технологии за RDF-изация на знание. Освен това беше направен избор на източници за информация и типове данни, които са интересни. Бяха разгледани основните модели данни и онтологии. След този анализ, публикувахме спецификацията на системата в \cite{senderov_open_2016} като отворен проект за дисертация (PhD project plan). По време на имплементацията обаче се оказа, че първоначалният план не отговаря на изменящите се изисквания на системата и на новите предизвикателства, които възникваха по време на имплементацията. По тази причина през втората и третата година на усилията за създаване на OpenBiodiv преминахме от модела на разработка waterfall, където след първоначален етап на изготвяне на софтуерната архитектура се преминава към продължителен етап на имплементация и тестване към модела agile, където спецификацията е разбита на по-малки user-stories, които биват реализирани ad-hoc в рамките на едно- или двумесечни спринтове. Например, за разработването на софтуерната библиотека за RDF-изация RDF4R, user-stories са ``възможност за работа с литерали'', ``възможност за импортиране през API endpoint'' и т.н. Първоначалният страх, че това ще доведе до ``хакерски код'' и недобре организирана софтуерна архитектура се оказа неоправдан, защото при изолирането на проблемите един от друг, успявахме да се концентираме върху проблемите последователно и да ги решаваме по елегантен начин. Аспекти на методологията agile, от които не успяхме да се възползваме напълно бяха колаборативните аспекти, може би поради факта, че системата беше разработена основно от кандидата със съдействието на един от програмистите на Пенсофт; в класическия смисъл на понятието agile team не съществуваше. По тази причина не практикувахме повечето ритуали като stand-ups и retrospection, а основно се концентирахме върху интеративния характер на разработката на софтуер.

Нашата визия за бъдещето на системата е работата по нея да се поеме от agile team, който да се възползва от пълния арсенал на методологията.